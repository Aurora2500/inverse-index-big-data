\documentclass{article}

\title{Inverted Index}

\newcommand{\authorblock}[1]{\begin{tabular}{@{}c@{}}#1\end{tabular}}

\author{\centering\begin{tabular}{ccc}
	\authorblock{
		Aurora Zuoris\\
		\normalsize{aurora.zuoris101@alu.ulgpc.es}
	} &
	\authorblock{
	Alejandra Ruiz de Adana Fleitas\\
	\normalsize{alejandra.ruiz104@alu.ulpgc.es}
	} \\ \\
	\authorblock{
	Lam Truong Nguyen\\
	\normalsize{lam.nguyen101@alu.ulpgc.es}
	} &
	\authorblock{
	Aris Vazdekis Soria\\
	\normalsize{aris.vazdekis101@alu.ulpgc.es}
	} \\ \\
	\authorblock{
	Jaime Ballesteros Domínguez\\
	\normalsize{jaime.ballesteros101@alu.ulpgc.es}
	} &
	\authorblock{
	Anna Barbara Król\\
	\normalsize{anna.krol101@alu.ulpgc.es}
	}
\end{tabular}}

\begin{document}
\maketitle
\abstract{
In this project, we undertook the task of creating an inverted index in Python to enhance search efficiency among a wide variety of text documents.
The primary challenge we faced was the need to quickly access documents containing specific terms within an extensive library.
To accomplish this task, we followed these steps: we collected a hundred English books, tokenized them to split them into words,
built an inverted index that could relate these terms to the specific ID of each book, and developed a series of tables that allowed
us to associate both the tokenized words with the IDs and these IDs with the metadata specific to each book.
}
\end{document}